\chapter{Conclusions and Future Work}
In this report, we presented Explorant, a novel onboarding and code exploration tool. Explorant was designed to improve many of the issues that we encountered in our case study of \texttt{malloc} and we think that it has succeeded at its mission and more. 

As time goes on, we hope that Explorant will continue to develop and address some of its main weaknesses. While many of the limitations brought up in section \ref{sec:design-limitations} cannot be fixed (like JIT support), a number of them can. The areas we think are ripe for future work are:
\begin{itemize}
    \item A user study evaluating the effectiveness of this tool and its usefulness
\item The ability to compare multiple traces at once and build a combined graph from all of them.
\item The ability to drag around nodes and add custom edges and labels within the graph so that the graphs can serve as official documentation
\item Fixing difficultiies with long running executions using complex function call heatmaps or simply limiting the execution to a smaller range (perhaps both) 
\item Automaic segmentation of events based off of filesystem hierarchies rather than modules. We think this would be particularly useful for codebases with a large number of disjoint files.
\item Possible automatic event definition based on some criteria from the trace that contain the fewest nodes but capture the most important flow paths. 
\end{itemize}

We strongly believe this tool can serve as a critical resource not only for new developers understanding a codebase, but also for senior engineers who can add Explorant annotations and have the junior engineers explore on their own. We will continue supporting Explorant and we hope others embrace and extend our work to realize the next generation of code exploration.

