\chapter{Understanding Code Onboarding}
% Explain why not using video lectures or explanations and say that it increases the number of probjects for why it is useful. 
% Having those lectures. 
% mention in the conclusion that it is hard to write documentation but that we are also using annotations as documentations. Circular.


In order to design Explorant as well as possible, we conducted a case study wherein we spent a week studying and understanding a code base while recording our observations about our onboarding process. The codebase that we chose was the glibc \cite{glibc} memory allocator, commonly referred to by its most used function, \texttt{malloc}. \texttt{malloc} uses a wide array of complicated datatypes, intricate macros, and C-style memory optimizations. The goal of this case study was to use only the codebase itself and no video lectures or online explanations. This was done so that we might arrive at a better understanding of what it is like to onboard with codebases that have less documentation. However, we will use many detailed source code comments in \texttt{malloc} as a stand-in for internal documentation or a mentor who knows the codebase well. 
% To design Explorant, we first conducted a case-study where we spent a week attempting to understand the glibc memory allocator (\begin{tt}malloc\end{tt}) using popular methodologies (source + complex IDE features + gdb + diagrams). We chose to study \begin{tt}malloc\end{tt} because of its complex yet well-thought-out design. During this process, we identified some of the biggest areas for improvement: the reality that debuggers are too implementation-focused for large-scale code flow understanding, the difficulty of maintaining detailed and up-to-date documentation, the mental burden of skimming error handling code, and the difficultiy of misunderstanding how large chunks of code work together. 

% clarify this sentence quite a bit. 
% swap to write code first then step through 
% then iterate 
% it is a time consuming laborious process that combines 


To tackle our \texttt{malloc} case study, we first researched tools for code onboarding. Almost all resources we could find online recommended the same advice: read the source code (with tools that provide features like jumping to definitions and collapsing modules), write some simple code to explore the behavior of the underlying system, step through the code with a debugger while keeping notes and building diagrams, and repeat. \cite{onboarding-case-study, onboarding-ramp}. This strategy worked well for us during our case study. 

\noindent From our study, we made the following observations:
\begin{enumerate}
        % remove signal to noise ratio. explain. 
    \item GDB is perfect for examining a single function execution in high detail, however, it fails at helping the programmer easily understand how functions fit together. In order to understand how functions fit together, we had to set many breakpoints. This allowed us to gain an in-depth understanding of a few functions however it did not always help with the larger picture. We also found that GDB showed large amounts of information that was mentally taxing to wade through and not very useful (like error handling code).
    \item \texttt{malloc}'s comments are extremely detailed and any changes that are made to \texttt{malloc} will require a great deal of attention to the documentation that is scattered around the source code. We didn't find any instances of incorrect comments but we noticed how certain data structures and ideas were referenced in comments throughout the program, yet they were only defined in a single location. We can easily envision how documentation that is separate from the implementation could become outdated and be hard to maintain.
        % rewrite point
    \item Manually writing down a lot of the internals of \texttt{malloc} was tiresome and distracting. This custom documentation was very useful to have because we referenced it later, though it interrupted our flow of thought and we found that we often wrote down many inconsequential implementation details that were unimportant.
    \item Jumping around the source code with definitions and code collapsing was critical to keep the amount of information in our ``working-memory'' focused on understanding an individual task.
        % Write kto explain this better
    \item After having completed our exploration, we thought we had a fairly deep understanding of \texttt{malloc}. However, once we started describing certain high-level interactions, we realized that we had misunderstood how multiple key components related to each other. By diving deep into certain areas, we had convinced ourselves that we understood the whole codebase at a similar depth which was clearly wrong.
\end{enumerate}

Most importantly, this case study validated our suspicion that there is room to improve this time-consuming and laborious process. 
