\frontmatter % Use roman page numbering style (i, ii, iii, iv...) for the pre-content pages

\pagestyle{plain} % Default to the plain heading style until the thesis style is called for the body content

%----------------------------------------------------------------------------------------
%	TITLE PAGE
%----------------------------------------------------------------------------------------

\begin{titlepage}
\begin{center}

\vspace*{.06\textheight}
{\scshape\LARGE \univname\par}\vspace{1.5cm} % University name
\textsc{\Large Doctoral Thesis}\\[0.5cm] % Thesis type

\HRule \\[0.4cm] % Horizontal line
{\huge \bfseries \ttitle\par}\vspace{0.4cm} % Thesis title
\HRule \\[1.5cm] % Horizontal line
 
\begin{minipage}[t]{0.4\textwidth}
\begin{flushleft} \large
\emph{Author:}\\
\href{http://www.johnsmith.com}{\authorname} % Author name - remove the \href bracket to remove the link
\end{flushleft}
\end{minipage}
\begin{minipage}[t]{0.4\textwidth}
\begin{flushright} \large
\emph{Supervisor:} \\
\href{http://www.jamessmith.com}{\supname} % Supervisor name - remove the \href bracket to remove the link  
\end{flushright}
\end{minipage}\\[3cm]
 
\vfill

\large \textit{A thesis submitted in fulfillment of the requirements\\ for the degree of \degreename}\\[0.3cm] % University requirement text
\textit{in the}\\[0.4cm]
\groupname\\\deptname\\[2cm] % Research group name and department name
 
\vfill

{\large \today}\\[4cm] % Date
%\includegraphics{Logo} % University/department logo - uncomment to place it
 
\vfill
\end{center}
\end{titlepage}


%----------------------------------------------------------------------------------------
%	ABSTRACT PAGE
%----------------------------------------------------------------------------------------

\begin{abstract}
\addchaptertocentry{\abstractname} % Add the abstract to the table of contents
The Thesis Abstract is written here (and usually kept to just this page). The page is kept centered vertically so can expand into the blank space above the title too\ldots
\end{abstract}

%----------------------------------------------------------------------------------------
%	ACKNOWLEDGEMENTS
%----------------------------------------------------------------------------------------

\begin{acknowledgements}
\addchaptertocentry{\acknowledgementname} % Add the acknowledgements to the table of contents
The acknowledgments and the people to thank go here, don't forget to include your project advisor\ldots
\end{acknowledgements}

%----------------------------------------------------------------------------------------
%	LIST OF CONTENTS/FIGURES/TABLES PAGES
%----------------------------------------------------------------------------------------

\tableofcontents % Prints the main table of contents

\listoffigures % Prints the list of figures

\listoftables % Prints the list of tables

%----------------------------------------------------------------------------------------
%	ABBREVIATIONS
%----------------------------------------------------------------------------------------

\begin{abbreviations}{ll} % Include a list of abbreviations (a table of two columns)

\textbf{LAH} & \textbf{L}ist \textbf{A}bbreviations \textbf{H}ere\\
\textbf{WSF} & \textbf{W}hat (it) \textbf{S}tands \textbf{F}or\\

\end{abbreviations}

%----------------------------------------------------------------------------------------
%	PHYSICAL CONSTANTS/OTHER DEFINITIONS
%----------------------------------------------------------------------------------------

\begin{constants}{lr@{${}={}$}l} % The list of physical constants is a three column table

% The \SI{}{} command is provided by the siunitx package, see its documentation for instructions on how to use it

Speed of Light & $c_{0}$ & \SI{2.99792458e8}{\meter\per\second} (exact)\\
%Constant Name & $Symbol$ & $Constant Value$ with units\\

\end{constants}

%----------------------------------------------------------------------------------------
%	SYMBOLS
%----------------------------------------------------------------------------------------

\begin{symbols}{lll} % Include a list of Symbols (a three column table)

$a$ & distance & \si{\meter} \\
$P$ & power & \si{\watt} (\si{\joule\per\second}) \\
%Symbol & Name & Unit \\

\addlinespace % Gap to separate the Roman symbols from the Greek

$\omega$ & angular frequency & \si{\radian} \\

\end{symbols}

%----------------------------------------------------------------------------------------
%	DEDICATION
%----------------------------------------------------------------------------------------

\dedicatory{For/Dedicated to/To my\ldots} 

%----------------------------------------------------------------------------------------
%	THESIS CONTENT - CHAPTERS
%----------------------------------------------------------------------------------------

\mainmatter % Begin numeric (1,2,3...) page numbering

\pagestyle{thesis} % Return the page headers back to the "thesis" style

% Include the chapters of the thesis as separate files from the Chapters folder
% Uncomment the lines as you write the chapters

\include{Chapters/Chapter1}
%\include{Chapters/Chapter2} 
%\include{Chapters/Chapter3}
%\include{Chapters/Chapter4} 
%\include{Chapters/Chapter5} 

%----------------------------------------------------------------------------------------
%	THESIS CONTENT - APPENDICES
%----------------------------------------------------------------------------------------

\appendix % Cue to tell LaTeX that the following "chapters" are Appendices

% Include the appendices of the thesis as separate files from the Appendices folder
% Uncomment the lines as you write the Appendices

\include{Appendices/AppendixA}
%\include{Appendices/AppendixB}
%\include{Appendices/AppendixC}

%----------------------------------------------------------------------------------------
%	BIBLIOGRAPHY
%----------------------------------------------------------------------------------------

\printbibliography[heading=bibintoc]

%----------------------------------------------------------------------------------------


\section{Executive Summary}

\section{Introduction}
Introduce glibc memory allocator

\section{Background}
most improtant is problem definition 
define key tersma and ideas 

\subsection{Current tools for source code understanding}

\subsection{Need for progress in this domain}

\subsection{Idea progression}

\section{Methodology}

\subsection{Librr-rs}
\subsection{Query Framework}


\subsection{Glibc memory allocator case study}


\section{Druid}


\section{Case study: glibc memory allocator}

\section{Results}
\subsection{}
\section{Further work}
\section{Conclusion}

\section{Efficient Address Recording}

\subsection{The problem}
When I began my analysis of librr, I quickly noticed that singlestepping and continuing (interrupting the process) incurred a heavy performance penalty (~100-1000x pentalty depending on the workload). This is unacceptable for anything but the simplest of programs. As such, I decided to use code stomping to jump into an area that I control which can record addresses without having to interrupt the underlying process. 

\subsection{My implementation}


Added stomped code: 
\begin{verbatim}
{instructions that were stomped}
XCHG rsp (beginning of my stack)
PUSH (address of instruction that was skipped)
XCHG rsp (beginning of my stack)
\end{verbatim}

Setting up the stack:
I created two additional memory mapped regions. One is r-w which I call my stack, and the other is rxw which I call my trampoline segment. I then find instructions whose length is at least 5 bytes and then insert a near jump to the location in the trampoline segment. 

\subsection{Stack layout}
RSP xchange spot 
RAX xchange spot 
..
end reserved space 
..
end of stack


\subsection{Stack overflow protection}
One problem that this implementation has is that it wastes stack space and is risky because if we overflow the stack then the user will get a segfault that is neigh-impossible to debug. As such, I also developed stack overflow protection to trigger an interrupt 

Key pieces of information:
\begin{itemize}
    \item No code I write here can increment the x86 retired branch counter as otherwise there will be diversions from the recording. This means that I cannot just compare RSP with some value and then interrupt. 
    \item The 0xcc instruction triggers an interrupt for a debugger (it is how gdb places software breakpoints).
    \item The NOP (No-op) instruction is the same size (1 byte) 
\end{itemize}

\begin{verbatim}
{instructions that were stomped}
XCHG rsp (beginning of my stack)
XCHG rax (beginning of my stack+4)
PUSH (address of instruction that was skipped)
PUSH 0xcccccccc
POP 
MOV AL (address at the base of the stack) 
MOV (RIP+1) AL
NOP
XCHG rsp (beginning of my stack)
XCHG rax (beginning of my stack+4)
\end{verbatim}




% \end{document}  
